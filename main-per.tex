\documentclass{article}
\usepackage{Definitions}
\usepackage{amsmath}
\usepackage{amssymb}

\usepackage{xepersian}
\settextfont{HM XNiloofar}
\setdigitfont{HM XNiloofar}
\linespread{1.3}

\title{احتمال و آمار}
\author{}

\begin{document}
\maketitle
\tableofcontents
\newpage

\section{مقدمه و توضیحات حیاتی}
\part{مبانی احتمال}

\section{شروعی بر احتمال}
\subsection{دست‌گرمی!}
\subsection{سوالاتی تاریخی}


\section{تعبیرهای مختلف احتمال}
\subsection{تعبیر فراوانی}
\subsection{تعبیر ذهنی}
\subsection{اصول کولموگروف}
\subsubsection{سازگاری با دیدگاه فراوانی‌گرا}
\subsubsection{سازگاری با دیدگاه ذهنی}
\subsection{پیچیدگیِ پیوستار}


\section{متغیرهای تصادفی}
\subsection{میانگین}
\subsection{چگالی و توزیع احتمال}


\section{احتمال شرطی}
\subsection{سازگاری با دیدگاه ذهنی}
\subsection{مفهوم استقلال به کمک احتمال شرطی}
\subsection{چند قضیه مهم احتمال شرطی}
\subsection{استقلال متغیرهای تصادفی}
\subsection{توزیع و چگالی احتمال توأم}


\section{امید ریاضی}
\subsection{محاسبه‌ی امید ریاضی}
\subsection{خطی بودن امید ریاضی}

\section{واریانس}
\subsection{مفهوم واریانس}
\subsection{چند نامساوی مشهور به کمک واریانس}


\section{چند توزیع مشهور}
\subsection{توزیع برنولی}
\subsection{توزیع دوجمله‌ای (و چندجمله‌ای)}
\subsection{توزیع نمایی}
\subsection{توزیع پواسون}
\subsection{توزیع نرمال (گاوسی)}


\section{دو قضیه‌ی مهم احتمال}
\subsection{قانون اعداد بزرگ}
\subsection{قضیه‌ی حد مرکزی}


\part{مبانی آمار}


\end{document}
